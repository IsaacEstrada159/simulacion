\documentclass{article}
\usepackage[spanish]{babel}
\usepackage[numbers,sort&compress]{natbib}
\usepackage{listings}
\usepackage{graphicx}

\title{Teor\'{i}a de colas}
\author{Isaac Estrada Garc\'{i}a }

\begin{document}

\maketitle
 
\section{Objetivos}
Examinar los efectos en los tiempos de ejecuci\'{o}n de los diferentes ordenamientos cambian cuando se var\'{i}a el n\'{u}mero de n\'{u}cleos asignados al cluster, ulizando como caso pr\'{a}ctico n\'{u}meros primos y no- primos como datos de entrada en un vector descargados de https://primes.utm.edu/lists/small/millions

\section{Metodolog\'{i}a y Resultados}
Se hace uso del lenguaje de programaci\'{o}n Python para determinal el n\'{u}meros de nucleos del ordenador con el siguiente c\'{o}digo.

\begin{lstlisting}[language = python]
>>> from multiprocessing import cpu_count
>>> cpu_ont()
8
\end{lstlisting}

 El c\'{o}digo principal empieza con una funci\'{o}n para determinar los n\'{u}meros primos y no-primos as\'{i} como sus factores.

\begin{lstlisting}[language = python]
  def factor(n):
    if n < 4:
        return -1
    if n % 2 == 0:
        return 2
    for i in range(3, int(ceil(sqrt(n))), 2):
        if n % i == 0:
            return i
    return -1
\end{lstlisting}


Como siguiente paso se importa el archivo dataprimes.txt ubicado en el repositorio simulacion URL: https://github.com/IsaacEstrada159/simulacion/tree/master/p3 creando un vector llamado datos se paraleliza y se miden los tiempos de ejecuci\'{o}n con 10 replicas variando el n\'{u}cleos. Finalmente el c\'{o}digo imprime resultados descrimptivos de los tiempos de ejecuci\'{o}n.
 

La \ref{tabla} muestra los resultados obtenidos, en la cual se observa una disminuci\'{o} en la mediana del tiempo conforme aumenta el n\�{u}mero de nucleos

\begin{table}[h]
\begin{center}
\caption{Efectos del tiempo variando el nucleo orden original}
\label{tabla}
\begin{tabular}{c c c c c c c}
\hline
\textbf{Nucleo}&\textbf{M�n.}&\textbf{Media}&\textbf{M�x.}\\
\hline
1 &4.731&4.930&5.560\\
2 &2.448&2.537& 3.250\\
3 &2.401&2.496&3.274\\
4 &1.655&1.859&2.722\\
\hline
\end{tabular}
\end{center}
\end{table}






\section{Conclusi\'{o}n}



\bibliography{p2}
\bibliographystyle{plainnat}
 
\end{document}
