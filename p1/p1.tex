
\documentclass{article}

\usepackage[spanish]{babel}


\title{Simulaci\'{o}n practica del movimiento Browniano y examinaci\'{o}n de los efectos de las dimenciones en los tiempos de regreso al origen de una part\'{i}cula}
\author{Isaac Estrada Garc\'{i}a }

\begin{document}

\maketitle

\section{Introducci\'{o}n}
El movimiento Browniano es un modelo matem\'{a}tico de una part\'{i}cula que describe la danza aleatoria de las part\'{i}culas que se debe a la agitaci\'{o}n molecular en la que se hayan inmersas.

En este trabajo los objetivos principales son modelar sistem\'{a}ticamente el movimiento browniano de una part\'{i}cula de una a ocho dimensiones del espacio, as\'{i} como tambi\'{e}n examinar el tiempo de regreso al origen de la part\'{i}cula analizando su caminata pseudoaleatoria.

\section{Hip\'{o}tesis}
Es posible que la probabilidad sea nula conforme las dimensiones vayan aumentando, provocando que la part\'{i}cula no regrese al origen.
\section{Objetivos}
Simular el movimiento Browniano de una part\'{i}cula examinando los efectos de la dimensi\'{o}n en el tiempo de regreso al origen, as\'{i} como tambi\'{e}n la probabilidad para dimensiones de 1 a 8 en incrementos lineales de uno, variando el n\'{u}mero de pasos de la caminata como potencias de dos con exponentes de 5 a 10 en incrementos lineales de uno, con 50 repeticiones del experimento para cade combinaci\'{o}n. 
\section{Simulaci\'{o}n y resultados}

\section{Conclusiones}

\section{Anexo}

\end{document}
