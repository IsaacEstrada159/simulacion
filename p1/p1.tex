
\documentclass{article}
\usepackage{listings}
\usepackage[spanish]{babel}


\title{Simulaci\'{o}n practica del movimiento Browniano y examinaci\'{o}n de los efectos de las dimenciones en los tiempos de regreso al origen de una part\'{i}cula}
\author{Isaac Estrada Garc\'{i}a }

\begin{document}

\maketitle

\section{Introducci\'{o}n}
El movimiento Browniano es un modelo matem\'{a}tico de una part\'{i}cula que describe la ``danza'' aleatoria de las part\'{i}culas que se debe a la agitaci\'{o}n molecular en la que se hayan inmersas.

En este trabajo los objetivos principales son modelar sistem\'{a}ticamente el movimiento browniano de una part\'{i}cula de una a ocho dimensiones del espacio, as\'{i} como tambi\'{e}n examinar el tiempo de regreso al origen de la part\'{i}cula analizando su caminata pseudoaleatoria.

\section{Hip\'{o}tesis}
Es posible que la probabilidad sea nula conforme las dimensiones vayan aumentando, provocando que la part\'{i}cula no regrese al origen.
\section{Objetivos}
Simular el movimiento Browniano de una part\'{i}cula examinando los efectos de la dimensi\'{o}n en el tiempo de regreso al origen, as\'{i} como tambi\'{e}n la probabilidad para dimensiones de 1 a 8 en incrementos lineales de uno, variando el n\'{u}mero de pasos de la caminata como potencias de dos con exponentes de 5 a 10 en incrementos lineales de uno, con 50 repeticiones del experimento para cade combinaci\'{o}n. 
\section{Simulaci\'{o}n y resultados}

La simulaci\'{o}n del movimiento Browniano se realizo con lenguaje de programaci\'{o}n python que es el siguiente codigo.Este da como resultado los pasos que toma la particula al llegar al origen.

\begin{lstlisting}[language=python]

from random iport random

dim = n # 1 A 8 USANDO 3 DIMENSIONES PARA LOS RESULTADOS
largo = 1024
pasitos = 0
milis = 0
corridas = 50

def paso(pos, dim):
    d = randint(0, dim-1)
    pos[d]+= -1 if random() < 0.5 else 1
    return pos

    

def experimento(largo, dim, pasitos, milis):
    pos = [0] * dim
    for t in range(largo):
        pos = paso(pos, dim)
        pasitos = pasitos + 1
        if  all([p == 0 for p in pos]):
            milis = pasitos
            print(milis)
            pasitos = 0
            return milis
  

for replicas in range (corridas):
experimento(largo, dim, pasitos, milis)
print ("fin")

\end{lstlisting}

Resultados

fin
fin
fin
fin
822
fin
fin
fin
fin
2
fin
938
2
fin
fin
fin
fin
2
4
114
fin
fin
fin
6
2
10
16
fin
66
fin
fin
2
2
fin
fin
fin
fin
fin
fin
fin
fin
fin
fin
fin
100
fin
fin
fin
fin
fin
14
fin
fin
fin
fin
4
fin
fin
fin
fin
2
fin
2
fin
fin
fin
fin
fin
2
fin


\section{Conclusiones}
Cuando las dimenciones son menores de 1 a 4 la probalidad exs de 0.9 a 0.2 de pasar por el origen as\'{i} que el tiempo o paso por el origen es mayor, pero ariba de la dimencion 5 la probabilidas es casi nula y los pasos por el origen tambien lo son, por lo tanto se puede definir que el tiempo de regreso al origen es infinito. Para los resultados se uso 3 dimensiones obrervando que regresa al origen 20 veces con tiempos o pasas de 2 hasta 938.


\end{document}
